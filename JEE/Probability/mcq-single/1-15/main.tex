\iffalse

\title{ASSIGNMENT2 - PROBABILITY}
\author{EE24BTECH11012 - Bhavanisankar G S}
%\section{mcq-single}
\fi 
%\begin{enumerate}
	\item Two fair dice are tossed. Let \emph{x} be the event that the first die shows an even number and \emph{y} be the event that the second die shows an odd number. The two events \emph{x} and \emph{y} are : \hfill\brak{1979}
  \begin{enumerate}
\item Mutually exclusive \item Independent and mutually exclusive  \item Dependent   \item None of these
\end{enumerate}
\item Two events \emph{A} and \emph{B} have probabilities 0.25 and 0.50 respectively. The probability that both \emph{A} and \emph{B} occur simultaneously is 0.14 . Then the probability that neither A nor B occurs is \hfill\brak{1980}
\begin{enumerate}
\begin{multicols}{2}
\item 0.39  \item 0.25  \item 0.11  \item None of these
\end{multicols}
\end{enumerate}
\item The probability that an event \emph{A} happes in one trial of an experiment is 0.4 . Three independent trials of the experiment are performed. The probability that the event \emph{A} happens at least once is \hfill\brak{1980}
\begin{enumerate}
\begin{multicols}{2}
\item 0.936  \item 0.784  \item 0.904  \item None of these
\end{multicols}
\end{enumerate}
\item If \emph{A} and \emph{B} are two events such that $ P(A) >0 $, and $ P(B) \neq1 $, then $ P \brak{\frac{\vec{A}}{\vec{B}}}$ is equal to \hfill\brak{1982 - 2 Marks}
	\begin{enumerate}
			\begin{multicols}{2}
			\item $ 1-P \brak{\frac{A}{B}} $ \item $ 1 - P \brak{\frac{\vec{A}}{\vec{B}}} $ \item $\frac{1 - P( A \cup B )}{P(\vec{B})} $ \item $ \frac{P(\vec{A})}{P(\vec{B})} $
			\end{multicols}
	\end{enumerate}
\item Fifteen coupons are numbered 1,2,3 ...,15 respectively. Seven coupons are selected at random, one at a time with replacement. The probability that the largest number on the coupon is 9, is \hfill\brak{1983 - 1 Mark}
	\begin{enumerate}
			\begin{multicols}{2}
			\item $\brak{\frac{9}{16}}^6$
			\item $\brak{\frac{8}{15}}^7$
			\item $\brak{\frac{3}{5}}^7 $
			\item None of these
			\end{multicols}
	\end{enumerate}
\item Three identical dice are rolled. The probability that the same number will appear on both of them is \hfill\brak{1984 - 2 Marks}
	\begin{enumerate}
			\begin{multicols}{4}
			\item $\frac{1}{6}$ \item $\frac{1}{36}$ \item $\frac{1}{18}$ \item $\frac{3}{28}$
			\end{multicols}
	\end{enumerate}
\item A box contains 24 identical balls of which 12 are white and 12 are black. The balls are drawn at random, one at a time from the box with replacement. The probability that the white ball is drawn 4th time in the 7th draw is \hfill\brak{1984 - 2 Marks}
	\begin{enumerate}
			\begin{multicols}{4}
			\item $\frac{5}{64}$ \item $\frac{25}{32}$ \item $\frac{5}{32}$ \item $\frac{1}{2}$
			\end{multicols}
	\end{enumerate}
\item One hundred coins, each with a probability of \emph{p} of showing heads are tossed once. If $0<\emph{p}<1$ and the probability of heads showing on 50 coins is equal to that showing on 51 coins, then the value of \emph{p} is \hfill\brak{1988 - 2 Marks}
	\begin{enumerate}
			\begin{multicols}{4}
			\item $\frac{1}{2}$ \item $\frac{49}{101}$ \item $\frac{50}{101}$ \item $\frac{51}{101}$
			\end{multicols}
	\end{enumerate}
\item India plays two matches each with West Indies and Australia. In any match, the probabilities of India getting points 0,1 and 2 are 0.45, 0.05 and 0.50 respectively. Assuming that the outcomes are independent, the probability that India gets atleast 7 points is \hfill\brak{1992 - 2 Marks}
	\begin{enumerate}
			\begin{multicols}{4}
			\item 0.8750 \item 0.0875 \item 0.0625 \item 0.0250
			\end{multicols}
	\end{enumerate}
\item An unbiased die with faces marked 1,2,3,4,5 and 6 is rolled four times. Out of four face values obtained, the probability that the minimum face value is not less than 2 and the maximum face value is not greater than 5, is then: \hfill\brak{1993 - 1 Mark}
	\begin{enumerate}
			\begin{multicols}{4}
			\item $\frac{16}{81}$ \item $\frac{1}{81}$ \item $\frac{80}{81}$ \item $\frac{65}{81}$
			\end{multicols}
	\end{enumerate}
\item Let A,B,C be three mutually independent events.Consider the two statements $S_1$ and $S_2$. \textbf{$S_1:$} A and $ B \cup C $ are independent.\\ \textbf{$S_2:$} A and $ B \cap C $ are independent.\\ Then,  \hfill\brak{1994}
	\begin{enumerate}
		\item Both S1 and S2 are true \item Only S1 is true. \item Only S2 is true. \item Neither S1 nor S2 is true.
	\end{enumerate}
\item The probability that India winning a test match against West Indies is $\frac{1}{2}$. Assuming independence from match to match the probability that in a 5 match series, India's second win occurs at third test is \hfill\brak{1995S}
	\begin{enumerate}
			\begin{multicols}{4}
			\item $\frac{1}{8}$ \item $\frac{1}{4}$ \item $\frac{1}{2}$ \item $\frac{2}{3}$
			\end{multicols}
	\end{enumerate}
\item Three of the six vertices of a regular hexagon are chosen at random. The probability that the triangle with three vertices is equilateral, equals \hfill\brak{1995S}
	\begin{enumerate}
			\begin{multicols}{4}
			\item $\frac{1}{2}$ \item $\frac{1}{5}$ \item $\frac{1}{10}$ \item $\frac{1}{20}$
			\end{multicols}
	\end{enumerate}

\item For the three events A,B and C, $P(exactly one of the events A or B occurs) = P(exactly one of the two events B or C occurs) = P(exactly one of the two events A or C occurs ) = p$ and $P(all the three events occur simultaneously) = p^2$, where $ 0 < p < \frac{1}{2}$ . Then the probability of at least one of the three events A,B and C occuring is \hfill\brak{1996 - 2 Marks}
	\begin{enumerate}
			\begin{multicols}{2}
			\item $\frac{3p+2p^2}{2}$ \item $\frac{p+3p^2}{4}$ \item $\frac{p+3p^2}{2}$ \item $\frac{3p+2p^2}{4}$
			\end{multicols}
	\end{enumerate}
\item If the integers \emph{m} and \emph{n} are chosen at random from 1 to 100, then the probability that a number of the form $ 7^m + 7^n $ is divisible by 5 equals \hfill\brak{1999 - 2 Marks}
	\begin{enumerate}
			\begin{multicols}{4}
			\item $\frac{1}{4}$ \item $\frac{1}{7}$ \item $\frac{1}{8}$ \item $\frac{1}{49}$
			\end{multicols}
	\end{enumerate}
%\end{enumerate}
